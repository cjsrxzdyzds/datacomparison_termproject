\documentclass[11pt]{article}
\usepackage[utf8]{inputenc}
\usepackage[margin=1in]{geometry}
\usepackage{amsmath}
\usepackage{amssymb}
\usepackage{amsthm}
\usepackage{graphicx}
\usepackage{hyperref}
\usepackage{cite}
\usepackage{booktabs}
\usepackage{enumitem}

\title{CSCI 6351 Data Compression\\
\Large Term Project Proposal:\\
Comparative Analysis of Probability Models for Arithmetic Coding}
\author{Oscar Fang\\G42568236}
\date{\today}

\begin{document}

\maketitle

\section{Project Overview}

\subsection{What: Comparative Analysis of Probability Models for Arithmetic Coding}

This project will implement and compare different probability models used with arithmetic coding for data compression. Specifically, I will evaluate:

\begin{itemize}[noitemsep]
    \item \textbf{Markov Models:} 1st, 2nd, and 3rd order models that predict symbols based on previous context
    \item \textbf{Finite State Machine (FSM) Models:} State-based models that adapt to patterns like runs of symbols
    \item \textbf{Neural Network Models:} RNN/LSTM-based predictors that learn complex dependencies
\end{itemize}

Each model will be integrated with an arithmetic coder implemented in MATLAB and tested on diverse datasets including text files, binary data, and structured sequences (e.g., DNA).

\subsection{Why: Motivation and Objectives}

Arithmetic coding can achieve near-optimal compression when paired with an accurate probability model. However, different models have different strengths:

\begin{itemize}[noitemsep]
    \item Simple models (1st order Markov) are fast but may miss patterns
    \item Complex models (3rd order, neural networks) can capture more dependencies but require more memory and computation
    \item Specialized models (FSM) may excel on specific data types
\end{itemize}

\textbf{Key Questions:}
\begin{enumerate}[noitemsep]
    \item How does model complexity affect compression ratio and speed?
    \item Which models work best for different data types?
    \item What are the practical trade-offs between compression quality and computational cost?
\end{enumerate}

\textbf{Expected Outcomes:}
\begin{itemize}[noitemsep]
    \item Quantitative comparison of compression performance across models and datasets
    \item Analysis of memory usage, encoding/decoding time, and compression ratios
    \item Practical guidelines for choosing models based on application requirements
    \item Open-source MATLAB implementation for educational use
\end{itemize}

\subsection{How: Implementation Approach}

All models will be implemented in MATLAB with a common arithmetic coding framework:

\begin{enumerate}[noitemsep]
    \item \textbf{Arithmetic Coder:} Base encoder/decoder with interval arithmetic and E1/E2/E3 scaling
    \item \textbf{Probability Models:} Each model provides probability distributions for the next symbol given context
    \item \textbf{Adaptive Updates:} Models update as data is processed (online learning)
\end{enumerate}

\textbf{Datasets:}
\begin{itemize}[noitemsep]
    \item Text: English text, source code (10KB - 1MB files)
    \item Binary: Executable files, images
    \item Structured: DNA sequences
\end{itemize}

\textbf{Evaluation Metrics:}
\begin{itemize}[noitemsep]
    \item Compression ratio (compressed size / original size)
    \item Encoding and decoding time
    \item Memory usage
    \item Comparison against baselines (Huffman, gzip, entropy)
\end{itemize}

\section{Timeline and Deliverables}

\subsection{Timeline}

\begin{table}[h]
\centering
\begin{tabular}{|l|l|}
\hline
\textbf{Week} & \textbf{Tasks} \\ \hline
1 & Implement arithmetic coder and 1st/2nd order Markov models \\ \hline
2 & Implement 3rd order Markov and FSM models \\ \hline
3 & Implement neural network models and run experiments \\ \hline
4 & Analyze results and write final report \\ \hline
\end{tabular}
\end{table}

\subsection{Deliverables}

\begin{enumerate}[noitemsep]
    \item Unified comparison of traditional and neural probability models for arithmetic coding
    \item Quantitative analysis of compression-complexity trade-offs
    \item Practical recommendations for model selection
    \item Educational MATLAB implementation with documentation
\end{enumerate}

\end{document}
