\subsection{Compression Performance}

\subsubsection{Text Data}
% Table comparing compression ratios for text data
\begin{table}[h]
\centering
% Table: Compression ratios for text data
\begin{tabular}{|l|c|c|c|c|c|}
\hline
\textbf{Model} & \textbf{alice29.txt} & \textbf{asyoulik.txt} & \textbf{lcet10.txt} & \textbf{plrabn12.txt} & \textbf{Average} \\ \hline
FSM & 2.17 & 2.11 & 2.21 & 2.29 & 2.20 \\ \hline
Markov-1 & 2.14 & 2.10 & 2.18 & 2.28 & 2.17 \\ \hline
Markov-2 & 2.05 & 1.91 & 2.28 & 2.32 & 2.14 \\ \hline
Markov-3 & 1.70 & 1.54 & 1.97 & 1.94 & 1.79 \\ \hline
LSTM & 2.41 & 2.30 & 2.51 & 2.59 & 2.45 \\ \hline
\end{tabular}

\caption{Compression ratios for text data. Lower is better.}
\label{tab:text_compression}
\end{table}

% Analysis of text results
% Analysis of text results
Figure~\ref{fig:compression_performance} compares the compression efficiency (bits per symbol) across different models.
The 3rd Order Markov model achieved the best compression on text data, but at the cost of significantly higher computational complexity.
The RNN model showed competitive performance, outperforming the 1st Order Markov model and FSM on text data.

\begin{figure}[h]
\centering
\includegraphics[width=0.8\textwidth]{images/compression_performance.png}
\caption{Compression performance (Bits Per Symbol) across different models and datasets.}
\label{fig:compression_performance}
\end{figure}

\subsubsection{Binary Data}
% Similar structure for binary data
\begin{table}[h]
\centering
% Table: Compression ratios for binary/code data
\begin{tabular}{|l|c|c|c|c|c|c|}
\hline
\textbf{Model} & \textbf{kennedy.xls} & \textbf{cp.html} & \textbf{fields.c} & \textbf{grammar.lsp} & \textbf{xargs.1} & \textbf{Average} \\ \hline
FSM & 3.11 & 1.73 & 1.68 & 1.50 & 1.42 & 1.89 \\ \hline
Markov-1 & 2.69 & 1.73 & 1.70 & 1.52 & 1.43 & 1.81 \\ \hline
Markov-2 & 3.78 & 1.55 & 1.51 & 1.35 & 1.23 & 1.88 \\ \hline
Markov-3 & 2.93 & 1.42 & 1.35 & 1.25 & 1.15 & 1.62 \\ \hline
LSTM & 8.72 & 1.69 & 1.60 & 1.56 & 1.48 & 3.01 \\ \hline
\end{tabular}

\caption{Compression ratios for binary data.}
\label{tab:binary_compression}
\end{table}

\subsubsection{Structured Sequences}
% DNA and protein sequence results
\begin{table}[h]
\centering
% Table: Compression ratios for sequence data
\begin{tabular}{|l|c|c|c|}
\hline
\textbf{Model} & \textbf{sum} & \textbf{ptt5} & \textbf{Average} \\ \hline
FSM & 1.77 & 8.88 & 5.33 \\ \hline
Markov-1 & 1.76 & 8.77 & 5.27 \\ \hline
Markov-2 & 1.52 & 7.49 & 4.50 \\ \hline
Markov-3 & 1.36 & 6.49 & 3.92 \\ \hline
RNN & 2.00 & 10.24 & 6.12 \\ \hline
\end{tabular}

\caption{Compression ratios for DNA and protein sequences.}
\label{tab:sequence_compression}
\end{table}
