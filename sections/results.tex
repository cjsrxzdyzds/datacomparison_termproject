\section{Results}
\label{sec:results}

This section presents our experimental findings across different probability models and datasets.

\subsection{Phase 1 Verification}
We verified the correctness of our arithmetic coding implementation and basic Markov models using a short test string ("hello world! this is a test string for arithmetic coding."). The results confirm lossless compression and expected behavior for adaptive models.

\begin{table}[h]
\centering
\begin{tabular}{lccc}
\toprule
\textbf{Model} & \textbf{Original Bits} & \textbf{Compressed Bits} & \textbf{Bits Per Symbol} \\
\midrule
1st Order Markov & 456 & 447 & 7.84 \\
2nd Order Markov & 456 & 455 & 7.98 \\
\bottomrule
\end{tabular}
\caption{Phase 1 verification results on a 57-byte test string.}
\label{tab:phase1_verification}
\end{table}

Both models achieved a compression ratio slightly below 8 bits/symbol, which is expected for such a short string where the overhead of adaptive model initialization (Laplace smoothing) is significant relative to the data size. The 1st order model performed slightly better than the 2nd order model in this limited context, likely due to the sparsity of the 2nd order context in a short sequence.

\subsection{Phase 2 Verification}
We extended our verification to the advanced probability models implemented in Phase 2: 3rd Order Markov, Finite State Machine (FSM), and Long Short-Term Memory (LSTM) network.

\subsubsection{High-Order Markov and Neural Models}
We tested the 3rd Order Markov and LSTM models on the same test string as Phase 1.

\begin{table}[h]
\centering
\begin{tabular}{lccc}
\toprule
\textbf{Model} & \textbf{Original Bits} & \textbf{Compressed Bits} & \textbf{Bits Per Symbol} \\
\midrule
3rd Order Markov & 456 & 457 & 8.02 \\
LSTM (Neural) & 456 & 451 & 7.91 \\
\bottomrule
\end{tabular}
\caption{Phase 2 verification results on a 57-byte test string.}
\label{tab:phase2_verification}
\end{table}

The LSTM model achieved a compression ratio of 7.91 bps on this short string. While slightly higher than the previous simple RNN (7.77 bps) due to the increased parameter count (gates) requiring more data to adapt, the LSTM architecture provides significantly better capacity for modeling long-term dependencies in larger files. The 3rd Order Markov model performed similarly (8.02 bps), highlighting the difficulty of compressing very short strings with complex adaptive models.

\subsubsection{FSM Model (Run-Length)}
To verify the FSM model's ability to handle run-length sequences, we tested it on a synthetic string containing repeated characters ("AAAAABBBBB...").

\begin{table}[h]
\centering
\begin{tabular}{lccc}
\toprule
\textbf{Model} & \textbf{Original Bits} & \textbf{Compressed Bits} & \textbf{Bits Per Symbol} \\
\midrule
FSM Model & 240 & 226 & 7.53 \\
\bottomrule
\end{tabular}
\caption{FSM verification on a run-length sequence (30 symbols).}
\label{tab:fsm_verification}
\end{table}

The FSM model successfully compressed the run-length sequence to 7.53 bits/symbol, confirming its effectiveness in detecting and exploiting repeated patterns.

\subsection{Compression Performance}

\subsubsection{Text Data}
% Table comparing compression ratios for text data
\begin{table}[h]
\centering
% Table: Compression ratios for text data
\begin{tabular}{|l|c|c|c|c|c|}
\hline
\textbf{Model} & \textbf{alice29.txt} & \textbf{asyoulik.txt} & \textbf{lcet10.txt} & \textbf{plrabn12.txt} & \textbf{Average} \\ \hline
FSM & 2.17 & 2.11 & 2.21 & 2.29 & 2.20 \\ \hline
Markov-1 & 2.14 & 2.10 & 2.18 & 2.28 & 2.17 \\ \hline
Markov-2 & 2.05 & 1.91 & 2.28 & 2.32 & 2.14 \\ \hline
Markov-3 & 1.70 & 1.54 & 1.97 & 1.94 & 1.79 \\ \hline
LSTM & 2.41 & 2.30 & 2.51 & 2.59 & 2.45 \\ \hline
\end{tabular}

\caption{Compression ratios for text data. Lower is better.}
\label{tab:text_compression}
\end{table}

% Analysis of text results
% Analysis of text results
Figure~\ref{fig:compression_performance} compares the compression efficiency (bits per symbol) across different models.
The 3rd Order Markov model achieved the best compression on text data, but at the cost of significantly higher computational complexity.
The RNN model showed competitive performance, outperforming the 1st Order Markov model and FSM on text data.

\begin{figure}[h]
\centering
\includegraphics[width=0.8\textwidth]{images/compression_performance.png}
\caption{Compression performance (Bits Per Symbol) across different models and datasets.}
\label{fig:compression_performance}
\end{figure}

\subsubsection{Binary Data}
% Similar structure for binary data
\begin{table}[h]
\centering
% Table: Compression ratios for binary/code data
\begin{tabular}{|l|c|c|c|c|c|c|}
\hline
\textbf{Model} & \textbf{kennedy.xls} & \textbf{cp.html} & \textbf{fields.c} & \textbf{grammar.lsp} & \textbf{xargs.1} & \textbf{Average} \\ \hline
FSM & 3.11 & 1.73 & 1.68 & 1.50 & 1.42 & 1.89 \\ \hline
Markov-1 & 2.69 & 1.73 & 1.70 & 1.52 & 1.43 & 1.81 \\ \hline
Markov-2 & 3.78 & 1.55 & 1.51 & 1.35 & 1.23 & 1.88 \\ \hline
Markov-3 & 2.93 & 1.42 & 1.35 & 1.25 & 1.15 & 1.62 \\ \hline
LSTM & 8.72 & 1.69 & 1.60 & 1.56 & 1.48 & 3.01 \\ \hline
\end{tabular}

\caption{Compression ratios for binary data.}
\label{tab:binary_compression}
\end{table}

\subsubsection{Structured Sequences}
% DNA and protein sequence results
\begin{table}[h]
\centering
% Table: Compression ratios for sequence data
\begin{tabular}{|l|c|c|c|}
\hline
\textbf{Model} & \textbf{sum} & \textbf{ptt5} & \textbf{Average} \\ \hline
FSM & 1.77 & 8.88 & 5.33 \\ \hline
Markov-1 & 1.76 & 8.77 & 5.27 \\ \hline
Markov-2 & 1.52 & 7.49 & 4.50 \\ \hline
Markov-3 & 1.36 & 6.49 & 3.92 \\ \hline
RNN & 2.00 & 10.24 & 6.12 \\ \hline
\end{tabular}

\caption{Compression ratios for DNA and protein sequences.}
\label{tab:sequence_compression}
\end{table}

\subsection{Computational Performance}

\subsubsection{Encoding and Decoding Time}
\begin{figure}[h]
\centering
\includegraphics[width=0.8\textwidth]{images/encoding_time.png}
\caption{Encoding time comparison across models.}
\label{fig:encoding_time}
\end{figure}

% Analysis
The encoding time results (Figure~\ref{fig:encoding_time}) highlight the trade-off between model complexity and speed.
The 1st Order Markov and FSM models were the fastest, while the 3rd Order Markov and RNN models required significantly more time due to their complex state updates and, in the case of RNN, gradient descent steps.
Notably, the sparse 3rd Order Markov model was slower than the RNN on some datasets, likely due to hash map overheads in MATLAB.

\subsubsection{Memory Usage}
\begin{table}[h]
\centering
% Table: Peak memory usage for different models
\begin{tabular}{|l|c|c|c|}
\hline
\textbf{Model} & \textbf{Text} & \textbf{Binary} & \textbf{Sequences} \\ \hline
Markov Order-1 & XX MB & XX MB & XX MB \\ \hline
Markov Order-2 & XX MB & XX MB & XX MB \\ \hline
Markov Order-3 & XX MB & XX MB & XX MB \\ \hline
FSM & XX MB & XX MB & XX MB \\ \hline
RNN & XX MB & XX MB & XX MB \\ \hline
LSTM & XX MB & XX MB & XX MB \\ \hline
\end{tabular}

\caption{Peak memory usage (MB) for different models.}
\label{tab:memory_usage}
\end{table}

\subsection{Comparison with Baselines}
% Compare against Huffman, gzip, bzip2
% TODO: Uncomment and add figure when results are available
% \begin{figure}[h]
% \centering
% \includegraphics[width=0.8\textwidth]{baseline_comparison.pdf}
% \caption{Compression ratio comparison with standard compressors.}
% \label{fig:baseline_comparison}
% \end{figure}

\subsection{Model-Specific Findings}

\subsubsection{Markov Models}
[Key observations about how order affects performance]

\subsubsection{FSM Models}
[Situations where FSM excels or underperforms]

\subsubsection{Neural Models}
[Trade-offs between learning capacity and computational cost]


\subsection{Scaling Analysis}
To better understand how our models perform as data size increases, we analyzed the relationship between file size and both compression ratio and encoding time.

\begin{figure}[h]
\centering
\includegraphics[width=0.8\textwidth]{images/size_vs_ratio.png}
\caption{Impact of file size on compression ratio. Note the log scale on the x-axis.}
\label{fig:size_vs_ratio}
\end{figure}

Figure~\ref{fig:size_vs_ratio} shows that compression ratios generally improve (decrease) as file size increases, particularly for adaptive models like the 3rd Order Markov, which benefit from observing more data to build accurate probability tables.

\begin{figure}[h]
\centering
\includegraphics[width=0.8\textwidth]{images/size_vs_time.png}
\caption{Encoding time vs. file size (log-log scale).}
\label{fig:size_vs_time}
\end{figure}

Figure~\ref{fig:size_vs_time} illustrates the time complexity. The linear relationship on the log-log plot confirms that most models scale linearly with input size ($O(N)$), but with vastly different constants. The RNN and 3rd Order Markov models show a much steeper intercept, indicating high per-symbol processing costs.
